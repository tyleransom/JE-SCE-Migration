\begin{table}[ht]
\caption{Median Willingness to Pay by Attribute and Chosen Level of Attribute}
\label{tab:SortingTable}
\centering
\begin{threeparttable}
\begin{tabular}{lcc}
\toprule 
 & \multicolumn{2}{c}{Existing Amount of Attribute} \\
\cmidrule{2-3} 
Attribute & Low  & High  \\
\midrule 
Housing costs           &    -1,490 &    -1,213 \\ 
Crime                   &    -9,298 &    -9,138  \\ 
Family nearby           &    14,365 &    22,751*  \\ 
Square footage          &     4,338 &     1,122  \\ 
Financial moving costs (\%)  &     -5.14 &     -1.72*  \\ 
Taxes                   &    -5,951 &    -3,922  \\ 
Norms                   &     4,987 &     3,160  \\ 
School quality          &     1,706 &     5,442*  \\ 
Same residence          &       399 &     1,155  \\ 
Nonpecuniary moving cost     &   -55,281 &   -19,449*  \\ 
\bottomrule 
\end{tabular} 
\footnotesize{Note: Sample size differs across rows but removes never-movers and those with very small or negative income elasticities (36\% of the full =2,110$ sample). * indicates that the median difference is significant at the 5 percent level. Significance is based on bootstrapped standard errors. 
 
A high amount of the existing attribute refers to having an amount above the median for the following attributes: housing costs, crime, square footage, and taxes. For family, it refers to living within 50 miles of a family member. For financial moving costs, it refers to being in the top half of the income distribution. For norms, it refers to reporting the current agreeableness of norms as very or extremely. For school quality, it refers to having a child in the home under the age of 18. For  same residence, it refers to owning a home. For nonpecuniar moving costs, it refers to having an unconditional future move probability of 10\% or higher. }
\end{threeparttable} 
\end{table} 
