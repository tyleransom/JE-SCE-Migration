\begin{table}[ht]
\caption{Characteristics of Ever- and Never-Movers}
\label{tab:descEverNeverMover}
\centering
\begin{threeparttable}
\begin{tabular}{lccc}
\toprule 
Variable & Ever-Mover & Never-Mover & Total \\
\midrule 
Female                                    & 0.48 & 0.51* & 0.48 \\ 
White                                     & 0.77 & 0.80* & 0.77 \\ 
Age                                       & 50.96 & 58.19* & 51.93 \\ 
Married                                   & 0.63 & 0.58* & 0.62 \\ 
Lives with children                       & 0.43 & 0.39 & 0.42 \\ 
College graduate                          & 0.36 & 0.25* & 0.35 \\ 
Owns home                                 & 0.70 & 0.74 & 0.70 \\ 
Income (\$1000)                          & 78.51 & 67.46* & 77.02 \\ 
Pr(move) in next two years                & 0.30 & 0.11* & 0.27 \\ 
Moved during previous year                & 0.16 & 0.07* & 0.15 \\ 
Years lived in current residence          & 11.77 & 15.26* & 12.24 \\ 
Mobile                                    & 0.40 & 0.10* & 0.36 \\ 
Stuck                                     & 0.13 & 0.09* & 0.12 \\ 
Rooted                                    & 0.47 & 0.80* & 0.52 \\ 
\midrule 
Sample size                      & 1,861 &   249 & 2,110 \\ 
\bottomrule 
\end{tabular} 
\footnotesize{Source: Survey of Consumer Expectations collected in September 2018 and December 2019. 
 
\bigskip{} 
 
Notes: Never-mover refers to an individual who reported the same exact choice probability in every single scenario. * indicates significantly different from ever-movers at the 5\% level. Family proximity was only collected for the September and December waves. For further details, see Section \ref{sec:data} and notes to Table \ref{tab:descT1}.}
\end{threeparttable} 
\end{table} 
